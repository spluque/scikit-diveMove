\documentclass[tikz]{standalone}
\usepackage{tikz-3dplot}
\usetikzlibrary{angles,arrows,backgrounds}
\tikzstyle{background grid}=[draw, step=2mm, gray, very thin]

% Colors
\colorlet{fillcol}{black!25}     % fill color
\colorlet{veccol}{black!60}      % input vectors
% Reference frame axis styles
\tikzstyle{vec}=[-stealth,veccol,scale=2,line cap=round]
\tikzstyle{plane}=[very thin,fill=fillcol,draw=veccol,line cap=round]
\tikzstyle{rightangle}=[draw=veccol,very thin,angle radius=4mm,
pic text=.,pic text options={veccol},anchor=west]
\tikzstyle{pline}=[very thin,veccol,dash pattern=on 2pt off 1pt,
line cap=round]

\begin{document}

\tdplotsetmaincoords{70}{135}
\begin{tikzpicture}[tdplot_main_coords, scale=1.5, % show background grid,
  every node/.style={scale=0.4}, every circle/.style={radius=0.25pt}]
  % Macros for key quantities
  \pgfmathsetmacro{\psiangle}{50}
  \pgfmathsetmacro{\px}{1.25}     % P1 x component
  \pgfmathsetmacro{\py}{0}     % P1 y component
  \pgfmathsetmacro{\pz}{0}     % P1 z component
  % Coordinates
  \coordinate (O) at (0,0,0);
  \coordinate (IX) at (0.75,0,0);
  \coordinate (IY) at (0,0.75,0);
  \coordinate (IZ) at (0,0,0.75);
  \coordinate (P1) at (\px,\py,\pz);

  % Create rotated frame and rotated vector
  \tdplotsetrotatedcoords{\psiangle}{0}{0}
  % Get coordinates of P2 in main coordinate system
  \tdplottransformrotmain{\px}{\py}{\pz}
  \coordinate (P2) at (\tdplotresx,\tdplotresy,\tdplotresz);
  % Main frame
  \draw[vec,black] (O) -- (IX) node[anchor=south]{\(x\)};
  \draw[vec,black] (O) -- (IY) node[anchor=south]{\(y\)};
  \draw[vec,black] (O) -- (IZ) node[anchor=east]{\(z\)};
  % Points
  \filldraw[vec] (O) circle node [anchor=south west]{\(P_{0}\)}
  (P1) circle node [anchor=south]{\(P_{1}\)}
  (P2) circle node [anchor=south west]{\(P_{2}\)};
  \draw[pline] (O) -- (P1) (O) -- (P2);
\end{tikzpicture}

\end{document}


%%% Local Variables:
%%% mode: latex
%%% TeX-master: t
%%% End:
